\documentclass{article}

\usepackage{fouche}
\author{}
\date{}
\title{Attento a come parli!}
\begin{document}
\maketitle
\noindent
Fosco Loregian \hfill A-26
\begin{itemize}
\item Valutate se gli spunti offerti dall’attività possono essere proficuamente utilizzati in classe.
\item Esprimete un parere motivato sull’adeguatezza del percorso scelto in funzione degli obiettivi proposti dall’attività stessa.
\item Date anche una valutazione sui punti di forza e sui possibili punti critici (o negativi) che l’esecuzione dell’attività può evidenziare in una classe.
\end{itemize}
Tra i pro dell'attività: essa non richiede particolari materiali né il trasferimento in un laboratorio; è quindi un'esercitazione ``a basso costo''. Permette di familiarizzare con dei concetti basilari e trasversali (com'è ovvio, la logica elementare ha utilità generale: nelle altre scienze, nell'analisi di un testo, nella sua schematizzazione, e in diverse circostanze della vita reale -quindi anche per quegli studenti che poi non conseguiranno una formazione scientifica). Indubbiamente quindi l'attività può essere utilizzata con profitto e va a formare una pertinenza. E' ingegnosa la pensata di affidare a dei giochi la lezione: la prima attività in fase 3 infatti consiste nel completare un sudoku; si tratta di un esempio semplice che probabilmente i ragazzi hanno già incontrato, e che quindi non richiede particolari introduzioni e non è distante da strutture che i ragazzi hanno in mente. 

Tra i contro: l'argomento trattato non si appoggia a un oggetto concreto che si può costruire e ``portare a casa''. Ciò impedisce agli studenti di associare un ricordo preciso all'esperienza, se questa non viene caricata di un aspetto familiare. La competenza logico deduttiva infatti, quando correlata a proposizioni che riguardano oggetti e strutture matematiche astratte, cresce col tempo anche in un individuo già adulto che studia per una formazione scientifica; figuriamoci quanto tempo impiega uno studente giovane a familiarizzare con il calcolo proposizionale!

L'esempio delineato nell'attività 1 è lievemente artificiale (non esiste un contesto dove si debba risolvere un problema simile; e il difetto di non avere appigli a un problema concreto è grave quando si tratta di motivare questo laboratorio: almeno un ragazzo chiederà: ``a cosa serve riempire un sudoku?''), e altrettanto, la richiesta di enunciare proprietà di un quadrato già formato non è ben posta (rischia di diventare molto dispersivo analizzare una per una le proposizioni costruite dai ragazzi; ovviamente in un mondo perfetto è un esercizio utile, perché fa capire quali sottigliezze rendono il calcolo proposizionale uno strumento raffinato; dopo una decina di volte, la classe è sicuramente persa in quanto a coinvolgimento). Il fatto che una delle attività consista del riempire un sudoku ha dei lati negativi: l'enigmistica è vista come una cosa da vecchi.

Eventuali migliorie alla luce di questo ultimo difetto: il calcolo proposizionale è costruito a partire da un linguaggio relativamente ristretto: le lettere dell'alfabeto latino, e i simboli unari e binari $\land, \lor, \lnot, \Rightarrow$. Si può trasformare la deduzione in un ``gioco di carte'' dove lo studente è chiamato a costruire per mezzo dell'assemblaggio di questi mattoni fondamentali la traduzione di un asserto in linguaggio naturale. Un certo set di lettere, e simboli specifici per la congiunzione, la disgiunzione, la negazione e l'implicazione possono essere assemblati in delle formule ben formate.

Si può cogliere l'occasione per alcune piccole variazioni sul tema dei sudoku e dei quadrati magici (un quadrato magico non è restrittivo come un sudoku perché i numeri possono ripetersi):
\begin{itemize}
  \item esistono sudoku che riempiono coi numeri $1,..,n$ un quadrato $n\times n$, per $n < 9$? Per esempio, quanti sudoku $2\times 2$ esistono? E quanti $3\times 3$?
  \item Quanti quadrati magici esistono, invece, in dimensione $2$ e $3$?
  \item Il ``quadrato del sator'' è il diagramma
  \[
  \begin{smallmatrix}
    s&a&t&o&r\\
    a&r&e&p&o\\
    t&e&n&e&t\\
    o&p&e&r&a\\
    r&o&t&a&s
  \end{smallmatrix}  
  \]
  e ha la proprietà che presenta la stessa frase latina se letto da sinistra a destra e dall'alto in basso, da destra a sinistra e dall'alto in basso, da sinistra a destra dal basso in alto, etc. Sebbene non sia un sudoku fatto di lettere, si tratta comunque di un arrangiamento di simboli con una enorme simmetria. E' un esercizio probabilmente divertente  quello di determinare i quadrati del sator in dimensione $2,3,4$ (ce ne sono? Quanti?)
  \item In alcuni licei scientifici si fa un minimo di algebra lineare: lo studio dei quadrati magici è lo studio di certe matrici con proprietà di simmetria rispetto a righe, colonne e diagonali; il prodotto di matrici magiche è magico?
  \item Esiste la flebile possibilità di un collegamento interdisciplinare: quadrati magici appaiono nella \emph{Melencholia} di D\"urer.
\end{itemize}
La strategia è adeguata in funzione degli obiettivi: sì, ma si vedano i contro su esposti. Probabilmente senza un appiglio concreto per ogni esempio gli studenti si annoieranno. Il calcolo proposizionale è meglio affrontato per via algebrica, e la definizione di fbf può risultare abbastanza misteriosa senza dei prerequisiti (soprattutto, una certa sensibilità astrattiva).
\end{document}