\documentclass{article}


\usepackage{lmodern, hyphenat}
\title{Pedagogia nella prospettiva della personalizzazione}
\author{Fosco Loregian}
\usepackage{fouche}
\begin{document}
\maketitle 

Tra i tratti distintivi della scuola moderna c'è la capacità consegnare dei codici culturali adeguati ad una realtà sempre più multiculturale.

Nello spazio di una generazione gli strumenti per leggere correttamente gli eventi, sono mutati completamente; la propagazione delle notizie è immediata, molto spesso non-mediata, la loro contraffazione sempre più semplice, il confronto con culture diverse dalla nostra è ormai ineludibile (una modesta connessione internet rende possibile l'accesso all'intero scibile umano), e i mezzi di comunicazione presentano con sempre meno filtri questo confronto, spesso trasformandolo in scontro. La mancanza di mediazione nella costruzione delle notizie impone alla scuola il ruolo di organizzare in un flusso ordinato di conoscenze gli stimoli cui i ragazzi in età scolare sono esposti. 

La scuola costituisce uno degli ambienti privilegiati per decifrare la realtà, soprattutto a causa della presenza costante nella quotidianità dei ragazzi; in un mondo che è sempre più facilmente traversabile, deve porsi come obiettivo prioritario l'integrazione\fshyp{}superamento delle diversità, la loro armoniosa convivenza, il disinnesco dei tentativi di instillare odio per il diverso e per realtà che sono alternative -per estrazione familiare, culturale o economica- alla media statistica.

E' proprio nella sua capacità -nel suo dovere- di essere un luogo di educazione, oltre che di formazione, che si realizza questa necessità; nella fattispecie si possono ravvisare nel testo quei temi generali che dovrebbero ispirarla.
\begin{itemize}
  \item tutte le culture sono a portata di mano fin dai primi anni di scuola, ma manca uno strumento per navigare attraverso le nozioni; un ragazzo delle medie può fare facilmente una ricerca su qualsiasi argomento senza uscire di casa, ma gli manca lo spirito critico per organizzare coerentemente le informazioni.
  \item le barriere culturali si combattono curando l'analfabetismo (anche quello di ritorno): molte delle classi di oggi sono multi-etniche\fshyp{}culturali, e gli studenti sono abituati a interagire con il ``diverso'' ogni mattina. Il rischio è che poi tornino a casa, a contatto con realtà diverse, a volte intolleranti. La scuola deve agire sullo studente offrendogli un messaggio di fratellanza, ma deve proporlo in parte anche ai genitori, attraverso i figli e attraverso il sodalizio scuola-famiglia (è anche a questo che si riferisce lo sviluppo ``sia orizzontale che verticale'' menzionato nel testo). 
  
  Questo processo di integrazione deve fare riferimento agli studenti curandoli nelle loro specificità; in quanto tale deve porsi come mezzo per la determinazione della personalità di ciascuno, nella convivenza reciproca che si realizza dando strumenti concreti per la gestione del conflitto, definito come ``inevitabile'': è infatti evidente che in ogni gruppo di persone chiamate a crescere insieme si sviluppino simpatie e antipatie. Sta alla scuola proporre dei mezzi per superare queste difficoltà istintive o maturate nel tempo.
  \item ``E' decisiva una nuova alleanza tra scienza, storia, discipline umanistiche arti e tecnologia in grado di delineare la prospettiva di un \emph{nuovo umanesimo}'': molto spesso, purtroppo, il desiderio di trascendere una distinzione netta tra i saperi resta parola vuota. Qual è il motivo profondo per cui si fa così fatica a raggiungere una vera formazione interdisciplinare, che prenda la cultura umana (scienze, storia, storia del pensiero, letteratura e arte\dots) come un tutt'uno da indagare ora con un linguaggio ora con un altro?
\end{itemize}
Alla luce di queste linee guida, il rapporto tra docente e discente si dovrebbe orientare al superamento della ``lezione tradizionale'' cattedratica e rigida; ciò a favore di un metodo più flessibile organizzato per interdisciplinarietà o linee tematiche. 

Preso atto che la scuola perde il suo primato formativo ed educativo e deve affiancarsi ad altre realtà, come la famiglia e l'immersione nei mezzi di comunicazione, si deve orientare a un modo di insegnare che prepari gli studenti all'ingresso in società e professioni che sono ormai flessibili e meno definite; la scuola deve rispecchiare questo cambiamento, adottando una duplice strategia: da un lato, vanno formati dei ragazzi che, entrando nel mondo adulto, troveranno quei codici di lettura che la scuola ha consegnato loro; dall'altro, andrebbe creata una nuova generazione capace di immettere nel mondo degli adulti quelle buone idee e buoni propositi (convivenza civile, interesse per il diverso, consapevolezza della propria cultura in relazione ad altre\dots) di cui la scuola li ha dotati.
In linea generae, al di là delle competenze specifiche da costruire -che dovrebbero essere quanto più trasversali possibile, e quanto più possibile volte alla competenza pratica- lo studente deve essere educato a un senso di igiene sociale e personale; in poche parole a sentirsi parte della comunità cui lo si sta per fare entrare. Questo certamente fa il paio con l'obiettivo precedente, di costruire un senso di accettazione per la diversità.
\end{document}