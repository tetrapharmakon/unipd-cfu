\documentclass[dvipsnames]{amsart}
\usepackage{times}
\usepackage{tcolorbox}
\usepackage[italian]{babel}
\usepackage{ hyphenat, hyperref}

\usepackage[left=3.5cm,right=3.5cm]{geometry}
\tcbuselibrary{most}
\tcolorboxenvironment{quote}{blanker,
before skip=10pt,after skip=10pt,
borderline west={6mm}{5pt}{gray!15}}

\title{Antropologia filosofica}
\author{Fosco Loregian}
% \usepackage{fouche}
\begin{document}
\maketitle 
% \section*{L'antropologia filosofica contemporanea -- M.T. Pansera}
{\sc Nota.} Nel testo che segue mi riferisco a [P] come al testo in esame.
\section{}
\begin{itemize}
  \item \emph{discuta il passaggio dalla visione pre-moderna e alla concezione moderna della domanda sull'uomo, mettendo in evidenza tanto le evenutali linee di continuità, quanto le ragioni della discontinuità};
\end{itemize}
La frattura tra la concezione classica e quella moderna di uomo si attua completamente durante la prima metà del XX secolo, ma ha delle radici più antiche (per esempio, la concezione di \emph{homo dyonisiacus} è in netta controtendenza rispetto alla precedente idea di uomo, vuoi come essere timorato di Dio, nel primitivo concetto di \emph{homo religiosus}, vuoi come essere illuminato dal logos nel successivo concetto di \emph{homo faber}); la rottura col passato si compie quando l'immagine dell'uomo predicata dalle correnti positiviste viene meno, dopo la prima guerra mondiale, per essere poi completamente distrutta dopo la seconda: è difficile rimanere devoti all'inarrestabile progresso della Scienza vedendo l'effetto deturpante dell'iprite sulla pelle, lo \emph{shellsock} negli occhi di un soldato di trincea, o le centinaia di scheletri ammassati fuori da Auschwitz. 

L'uomo, che prima era timorato di Dio, fabbro del suo destino, preda dei suoi istinti, demiurgo del suo mondo, adesso è \emph{anche} quell'entità capace di auto-annientarsi nel miraggio di un Reich di mille anni, di sintetizzare un gas nervino, talmente disposto a uccidere da arrivare a compiere la suprema ingiuria etimologica di spezzare un atomo. Impossibile che questo non metta una certa fretta a rispondere al ``problema uomo''; se non altro per evitare l'estinzione.

Per certi versi, il distacco dai presupposti razionalisti, cartesiani è il germe del pensiero debole, che indubitabilmente andrà a influenzare il dibattito filosofico sul ``problema uomo''; tale distacco è se possibile avvertito con ancora maggiore forza alla luce delle ``tre ben note umiliazioni'' [P] inflitte dal progresso: dopo Copernico e Galileo, la Terra\fshyp{}l'uomo non è più al centro del cosmo; dopo Darwin, l'uomo non è più al pinnacolo della gerarchia dei viventi, ma è l'accidentale somma di mutazioni di una scimmia senziente; dopo Freud, l'uomo non è più padrone del proprio inconscio (e in effetti \emph{scopre di avere} un inconscio, consapevolezza affatto banale). 

Non senza una punta di velato sadismo, mi sento di continuare la lista delle umiliazioni: da una scrivania all'Università di Vienna, G\"odel spodesta l'ultimo tassello di razionalismo: il linguaggio genera mostri la cui verità o falsità è indecidibile all'interno del linguaggio stesso; per gli esegeti di Copenhagen la realtà a scale microscopiche è una entità granulare, indeterministica, sfumata a diventare un'onda di probabilità; per un ashkenazita fuggito a Princeton, il tempo è una materia talmente malleabile da riuscire a squagliare gli orologi, viaggiare veloci accorcia lo spazio e allunga il tempo\dots 

Tutte queste teorie, fondate dall'uomo per penetrare alcune pieghe dell'assoluto, non possono non offrire nuovi quesiti e nuove risposte al ``problema uomo''. Ed è così che nell'evoluzione a velocità super-esponenziale delle scienze (sia scienze pure che umane) l'antropologia trova un nuovo terreno fertile per fondarsi: oltre a quello filosofico, il punto di rottura maggiore col passato è determinato dalla volontà di coniugare il progresso scientifico ottenuto nei precedenti cinquant'anni con il problema, antico quanto l'uomo stesso, di determinare la sua natura e il suo posto nel mondo (il suo \emph{darsi alla realtà}, per usare una locuzione più filosofica). Ora, tale rottura è soprattutto posta in relazione alla necessità, col crollo del positivismo, di dare nuova speranza al progresso umano, dopo che le due guerre hanno lacerato e reso ingenua l'idea che la Scienza risolverà tutti i problemi, che le macchine solleveranno l'uomo dall'onere del lavoro, che siamo lì lì per penetrare gli ultimi segreti del linguaggio, della matematica e della fisica.

La nascita dell'antropologia filosofica appare quindi legata a doppio filo a tutte queste umiliazioni, dopo le quali però la filosofia non resta stesa al tappeto; il progresso scientifico aveva
\begin{quote}
  rappresentato senza dubbio un progresso, aveva[\dots\unkern] però messo in crisi l'uomo e la sua
autoimmagine, facendogli avvertire in modo più acuto il bisogno di un intervento filosofico per ricomporre in una visione unica i diversi brandelli in cui era stata smembrata la sua figura. \hspace{\fill} [P]
\end{quote}
Da ultimo, un motivo di rottura minore rispetto a quello già enunciato si trova tra l'antropologia dei fondatori e quella piu moderna: da uno stile molto piu attaccato alle preconcezioni, cioè a una pre-esistente visione filosofica (per esempio Scheler: l'uomo è quella entità animata da scintilla divina), si passa a una antropologia più a suo agio con il linguaggio scientifico, con la biologia, con l'etologia, e con l'antropologia stessa, ma fatta in una chiave più analitica e in una prospettiva culturale (linguistica, etnologica etc).
% Alcune osservazioni da espandere:
% \begin{itemize}
%   \item Confrontando quello di Pansera con altri testi, ho smussato questa opinione, quindi penso sia un problema dell'autrice -o una mia scarsa comprensione di alcune parti che addolciscono questa sensazione istintiva- ma la discussione mi sembra portata da un POV irrimediabilmente eurocentrico: [...]

%   E già che ci siamo, un altro commento sulla stessa linea (ossia un commento che trasli la discussione sul piano di una visione meno euro- e fallo-centrica): dove sono le donne in tutta questa disamina? Le teorie con cui veniamo a contatto sono state, per nove decimi dello span di tempo in cui si sono sviluppate, immerse in una società maschilista
% Non fatico a credere che ci sia stato uno sviluppo dell'antropologia nella prima ondata di femminismo. Dov'è l'eredità di questo momento?
%   \item Il tentativo di risolvere il ``problema uomo'' si riduce spesso all'analisi `uomo VS animale': cos'ha l'uomo che l'animale non ha? Cos'ha l'animale che l'uomo non ha? Questo mi pare riduttivo, e la limitatezza di questo POV mi sembra venire esattamente dalla biologia/zoologia/etologia di cui l'antropologia culturale sembra essere tanto fan.
%   \item La sensazione generale (spero sia permesso enunciare una opinione, sebbene poco informata) è che la disciplina sia intrisa di un velo ``umanistico, troppo umanistico'' per portare auna risposta oggettiva e concreta a proposito del ``problema uomo'': in 30 pagine un vero intreccio con il linguaggio scientifico io non l'ho proprio visto.
%   \item dov'è la scienza che viene tanto sbandierata come mezzo per affrancare l'antropologia filosofica dall'umanesimo? Vedo biologia e altri cazzi ma ben poche scienze dure. Da matematico quale sono, non vedo questa urgenza ``di un intervento filosofico per ricomporre in una visione unica i diversi brandelli in cui era stata smembrata la figura [umana]''. Vi sono strutture molto più urgenti e definitive da svelare.
%   \item Un motivo di rottura minore rispetto a quello già enunciato si trova tra l'antropologia dei fondatori e quella piu moderna: da uno stile molto piu attaccato alle preconcezioni, cioè a una pre-esistente visione filosofica (per esempio Scheler: l'uomo è quella cosa che ha scintilla divina), si passa a una antropologia più a suo agio con il linguaggio scientifico, con la biologia, con l'etologia, con l'antropologia stessa ma fatta in una chiave più analitica e in una prospettiva culturale (linguistica, etnologica etc).
% \end{itemize}
\section{}
\begin{itemize}
  \item \emph{analizzi criticamente alcune o tutte le posizioni teoriche, illustrate dall'autrice, alla luce del rapporto di tensione tra il paradigma ``natura'' e il paradigma ``cultura''} 
\end{itemize} 
Il linguaggio fa da tramite tra la natura (vista come impulso e istinto) e la cultura; il linguaggio è infatti il veicolo primario con cui si trasmette cultura, con cui possiamo tramandare, all'interno della stessa generazione, la maniera corretta di seppellire i morti, il modo di smembrare un alce, o il volto di Dio. Se la cultura si pensa anc\'ora come l'insieme di conoscenze, credenze, arte, morale, diritto, costume e qualsiasi altra capacità e abitudine acquisita dall'uomo in quanto membro di una società, da ciò si evince che non c'è cultura senza società; ma allora la società si può pensare come la rappresentazione concreta della rete di relazioni tra i suoi membri, in rapporto a un'altra contro la quale la nostra si determina per confronto o scontro.

Questa determinazione del sé e dell'altro attraverso le mutue differenze definisce la cultura, ed è possibile solamente attraverso il linguaggio. Di questo pensiero probabilmente banale sia la prova la varietà di termini per indicare un \emph{popolo} nella lingua greca: si può essere \emph{ethnos}, ``la mia tribù'' con cui condivido gli stessi modi di leggere la realtà; si può essere \emph{xenos}, la tribù dell'altro, che quindi non la vede come noi; si può essere \emph{demos}, il popolo che ha diritto di partecipare alle decisioni collettive (e non è scontato che tutto l'ethnos sia demos); chi non è nemmeno greco poi è \emph{barbaros}, quello che invece di avere un linguaggio balbetta, ed è definito da questa sua deficienza.

Questo apre la porta a un'analisi personale: da matematico, faccio quel che faccio pensando che la mia disciplina si possa definire come quel corpo di conoscenze che è il presupposto al linguaggio; perciò, sono estremamente suscettibile all'idea per cui l'uomo è quell'entità che ha linguaggio, che mediante il linguaggio determina la propria umanità. Si legge in [P], che a sua volta cita Gadamer, che ``il mondo è costituito mediante il linguaggio''; è vero. La libertà di determinare linguisticamente, oltre che esperienzialmente, la realtà che ci circonda è un tratto distintivo (\emph{il} tratto distintivo, a mio modo di vedere) della natura umana. Questo ruolo del linguaggio come tramite tra natura e cultura trova maggiore conferma quando si pensa che, sebbene l'uomo non sia l'unico animale a possedere un linguaggio, solo in quest'ultimo esso è uno strumento così preponderante nella costruzione della cultura. Fino al punto che, la concettualizzazione avvenendo solamente all'interno del linguaggio, è diventato difficile scindere la realtà dall'immagine linguistica che ne costruiamo.

Questo intero tema mi riporta alla mente un'intervista tra Heidegger e un buddhista tibetano (riesco a riportarne solo una traduzione in inglese dato che non conosco il tedesco: è una trascrizione il più letterale possibile): 
\begin{quote}
``The crucial experience of my thinking, and that means, at the same time, for Western philosophy the reflection on the history of Western thought has shown me that in previous thought one question never was raised: namely, the question of \emph{Being}. And this question is of importance because in Western thought we determine the nature of man in relation to Being, and man exists in correspondence ot Being. That is, by this correspondence, man is the being that has language. And as distinguished from Buddhist teachings, Western thought makes an essential distinction between man and other livingbeings: plants and animals. Man is distinguished by having a language, that is by existingin a knowing-relation to Being.'' \hspace{\fill} \cite{heideggo}
\end{quote}
Heidegger in questo passo è chiarissimo: ``secondo questa corrispondenza, \emph{l'uomo è l'essere che possiede un linguaggio}''. Tale intervento rende abbastanza evidente la sua posizione rispetto all'antropologia filosofica. 

Il confronto tra Heidegger e il monaco permette tra l'altro un ulteriore commento: se chiaramente gli autori di riferimento moderni sono tutti europei, è altresì vero che anche nell'introduzione storica iniziale tutta la discussione sul ``problema uomo'' mi sembra riportata da un punto di vista irrimediabilmente eurocentrico: mi viene naturale domandarmi cosa abbiano pensato altri popoli distanti dalla cultura europea della stessa questione. Per esempio, visto che dietro il confronto tra Heidegger e un buddhista è una sineddoche per il confronto tra due ontologie, cosa hanno pensato di questo problema l'induismo e il buddhismo. La prima è una vera e propria religione, che in quanto tale non può non essersi preoccupata del ``problema uomo'': già Heidegger fa notare che una cultura imbevuta tanto a fondo della credenza nella reincarnazione deve produrre una diversa filosofia; lo fa, ed è quindi ovvio che produca dei divesi presupposti filosofici con cui affrontare il problema dell'antropologia -e altrettanto fa la divisione della società in caste molto rigide, costumi distanti dai nostri in termini di restrizioni alimentari, e soprattutto in termini di rapporto uomo/animale-.

Quest'ultima osservazione sia un pretesto per un ulteriore commento critico: il tentativo di risolvere il ``problema uomo'' si riduce spesso all'analisi del suo rapporto con il mondo animale, sia nelle linee di rottura che in quelle di continuità: cos'ha l'uomo che l'animale non ha? Cos'ha l'animale che l'uomo non ha? Cosa li accomuna? Si spera da ciò di costruire una bozza di risposta al ``problema uomo''. A me questa strategia pare un po' ingenua, e certamente riduttiva: la limitatezza di questo approccio mi pare venire esattamente da discipline come la biologia, la zoologia e l'etologia con cui l'antropologia culturale sembra essere tanto desiderosa di collaborare.
\bibliography{refs}{}
\bibliographystyle{amsalpha}
\end{document}