\documentclass[dvipsnames]{amsart}
\usepackage{times}
\usepackage{tcolorbox}
\usepackage[italian]{babel}
\usepackage{ hyphenat, hyperref}

\usepackage[left=3.5cm,right=3.5cm]{geometry}
\tcbuselibrary{most}
\tcolorboxenvironment{quote}{blanker,
before skip=10pt,after skip=10pt,
borderline west={6mm}{5pt}{gray!15}}

\title{Antropologia filosofica}
\author{Fosco Loregian}

\begin{document}
\maketitle 

\begin{itemize}
  \item[$\bullet$] 
\end{itemize}

C'è questo vago religiosismo che mi fa un po' schifo ed è capzioso

Cosa significa ``scegliere sé stesso moralmente''? E' davvero questa la definizione migliore di morale che abbiamo?

Esempio dell'uccidere una persona per salvarne 5: cosa è giusto? E altri esperimenti mentali che dimostrano, se non altro, che questa presupposizione dell'esistenza di un bene assoluto è perlomeno ingenua (e chiama naif la posizione di Kant? Pfft)

La verità è che la morale è tutt'altro che una costante eterna della specie umana, e ogni tentativo di dire che gnogni gogni dobbiamo cercare comunque l'universale è destinato a cadere nel vuoto. Come si cerca un universale che non esiste?

Proposta: i sistemi morali sono particolari sistemi formali paraconsistenti (e in quanto tali, Aristotele molto probabilmente non ci aveva capito un cazzo: l'identificazione tra terzo escluso, non contraddizione e principio di identità fa piu danno che bene)

Modelli relazionali di paraconsistenza: ad una proposizione, si associa un set di valori di verità $Prop \to 2^{Truth}$

Workare su questo pezzo di Uriel:


  Che i sentimenti abbiano basi culturali e’ noto, non c’e’ bisogno di menzionare sentimenti che esistono solo in Giappone, come il Mottainai o il Wabi-sabi, per dimostrarlo. Il secondo non e’ nemmeno traducibile in italiano, ed esiste solo una traduzione in tedesco, come parola composta: mangellhaftfreude. Ma non essendo un sentimento diffuso in Germania, (tutto il contrario) la parola suona come una forzatura e l’ho sentita coniare ad un giapponese che vive a Düsseldorf per spiegarmi “wabi-sabi”. E non voglio menzionare Yūgen.

  Prendiamo per esempio la vendetta. La vendetta e’ stata, per secoli e secoli, un sentimento del tutto legittimo. E non sto scherzando: sebbene oggi sia diluito nel piu’ generico (e blando) “rivalsa”, la vendetta era un impulso che faceva parte del codice civile e penale, ed e’ rimasto alla base di molti codici religiosi. Se leggete la stessa Bibbia, Dio viene descritto come “vendicativo” piuttosto spesso.
  
  A tutt’oggi, in alcune societa’ la vedetta e’ ancora parte del codice penale: per esempio, in Iran se fate del male a qualcuno la famiglia ha diritto ad una vera e propria vendetta, cioe’ puo’ chiedere al giudice di poter causare lo stesso danno subito. Se avete sfregiato una persona con l’acido, per dire, potreste venire sfregiati con l’acido da qualche familiare della vittima, perche’ viene stabilita la vendetta come diritto.
  
  Questo sentimento , che e’ una spinta a causare lo stesso danno subito a chi ce lo ha inflitto, nelle societa’ piu’ moderne si vede sempre meno. Non soltanto ogni traccia di vendetta e’ stata rimossa dal codice penale, ed essa e’ stata cancellata dagli scopi della giustizia, ma anche nel quotidiano questo sentimento si e’ diluito sempre di piu’ , passando da una determinazione bellicosa, pubblica e spesso giurata , ad una generica determinazione ad un atteggiamento ostile se capitera’ occasione, altrimenti diventa semplicemente una determinazione ad essere meno collaborativi del dovuto, quando non a limitarsi al dovuto.
  
  Se esaminiamo la letteratura della vendetta nel corso dei secoli, sino all’abolizione del duello per vendicare un’offesa subita, vediamo subito che il sentimento di vendetta e’ in via di estinzione. Compare pochissimo nella narrativa popolare, e anche quando compare viene sempre deprecato: molto diverso rispetto ai tempi in cui vendicare una ferita, anche d’onore, era considerato un dovere, e chi vendicava duramente i torti subiti era visto con ammirazione.
  
  A determinare l’estinzione di questo sentimento, la riduzione della sua intensita’ e la sua uscita dal codice penale e civile prima, e poi dallo spettro dei comportamenti apprezzati sono stati molti fattori, dall’arrivo del codice penale alla disgregazione dei clan familiari, sino alla cessazione delle societa’ basate sulla reputazione, nelle quali la minaccia di vendetta (e l’efficacia delle vendette passate) erano parte della reputazione stessa, detta impropriamente “onore”.
  
  Ma il problema non sono le cause: voglio sottolineare il fatto che i sentimenti non sono costanti eterne della specie umana.
  
  Potrebbe succedere all’amore? Certo. Cosi’ come e’ quasi scomparso il sentimento di “onore”, e il suicidio per onore (nel senso cavalleresco del termine, come bushi-do ) viene considerato una follia. La stessa vergogna e il senso del pudore sono molto affievoliti, ed e’ successo in pochissime generazioni.
  
  Il punto non e’ che oggi ci si vergogna per cose diverse: il punto e’ che la vergogna come sentimento e’ riservata solo a casi estremi. Allo stesso modo, l’amore e’ diventato una cosa molto diversa, ed e’ morto cosi’ come e’ nato, cioe’ nella trasformazione della famiglia tradizionale.
  
  Se un paio di millenni fa in quasi tutta Europa non era richiesta l’opinione della sposa, e spesso nemmeno quella dello sposo, per configurare un matrimonio (che era visto in un sistema di clan), ancora oggi in India i “matrimoni d’amore” sono considerati una fortunata eccezione, e il massimo che si prendeva in considerazione era la felicita’ degli sposi. Questo significa che nessuno si poneva neppure la domanda “ma si amano”: la domanda era “sono felici insieme”, che non presuppone alcun sentimento personale , ma solo la riuscita della coppia.
  
  Certo per un occidentale romantico la coppia e’ felice solo se i componenti sono innamorati, ma in realta’ sarebbe come dire che una torta riesce bene solo se farina e uova provano dei sentimenti reciproci: in realta’ il fatto che una coppia sia felice non deriva dall’ “amore”, semmai l’amore e’ un’etichetta che abbiamo dato alla parte poco comprensibile della felicita’ della coppia.
  
  Faccio presente pero’ che in periodi passati questa felicita’ era attribuita piu’ al carattere degli sposi (“docile” la sposa, “giusto” il padre) che ai sentimenti reciproci: la societa’ non avrebbe mai accettato la storia dell’amore, quando si sapeva bene che i matrimoni fossero combinati. Al massimo l’amore sarebbe venuto dopo il matrimonio, quando gli sposi si fossero conosciuti meglio e avessero cominciato ad apprezzarsi. E funzionava.
  
  Dopo la vendetta, direi che i sentimenti in via di estinzione siano pudore e vergogna, mentre l’amore si sta trasformando in qualcosa di molto meno “assoluto”. Siamo passati all’amore che dura tutta la vita mentre tutto il resto sono “flirt”, ad un amore che puo’ venire iterato su piu’ partner nel tempo. Ci sono discussioni pubbliche su cose come “ma l’amore a prima vista esiste?” , cosa che non era soggetta a dubbi nel periodo in cui si comincio’ a parlare di Cupido.
  
  L’amore materno , poi, e’ in via di demolizione molto piu’ velocemente di quanto si pensi: sebbene si sia sempre detto che fosse un sentimento innato in ogni donna, e che ogni donna dovesse provare questo sentimento di fronte a qualsiasi pargolo, il problema e’ che sempre piu’ donne (quelle che si sentivano delle merde per non aver provato questo sentimento) hanno trovato la forza di confessare che questo sentimento non lo hanno mai provato.
  
  E qui arriva il secondo punto: moltissimi sentimenti sono stati descritti in maniera poco realistica, e in diversi modi:
  
  I sentimenti vengono dipinti come innati nella specie umana, sottintendendo che ogni membro di tale specie li provi. E’ possibile che solo alcuni membri della specie li provino. E’ possibile, cioe’, che moltissime persone non abbiano mai provato amore nonostante siano sposate, che tantissime madri non abbiano mai provato qualcosa come l’istinto materno, ed altro.
  I sentimenti vengono dipinti come resistenti al tempo. Al contrario, ci sono sentimenti che sono quasi scomparsi, come la vendetta, che si sono affievoliti o che si stanno trasformando, come vergogna e amore, ed e’ del tutto possibile che alcuni di questi sentimenti si estinguano in futuro.
  I sentimenti vengono definiti come intangibili alle vicende terrene: tutti vogliono pensare che , certo, in periodi diversi i sentimenti hanno prodotto comportamenti diversi e letterature diverse, ma alla fine erano sempre quelli e sono quelli ovunque.
  i sentimenti siano intangibili alle analisi e alla ragione, rientrando in un mondo che la realta’ materiale (e le scienze che investigano sulla realra’ stessa) non puo’ tangere.
  I soliti letterati oggi si staranno chiedendo se valga la pena vivere in una societa’ senza amore, compassione o altro, ma potrei rispondere semplicemente che in passato molti uomini avrebbero considerato impossibile vivere senza quel sentimento di ammirazione di se’ che era l’onore (ed era diffuso il suicidio quando la persona riteneva impossibile ammirare se’ medesimo, cioe’ riteneva compromesso l’onore) : oggi come oggi si ritiene che l’onore non sia nemmeno un sentimento ma un concetto culturale. Ma se leggiamo letteratura passata possiamo sicuramente riconoscerlo come sentimento: semplicemente oggi e’ estinto.
  
  La stessa vergogna , che puo’ spingere le persone al suicidio (come capita nel revenge porn) e’ molto ridimensionata, al punto che se uno psicologo si trova ad aiutare una persona soggetta a queste aggressioni, la sua prima reazione e’ di insegnare a gestire il sentimento di vergogna. Ma “gestire un sentimento” va contro la sua stessa definizione, perche’ ne implica la sua razionalizzazione, la sua trasformazione in teoria logica. La sua qualita’ imperativa e’ quasi scomparsa.
  
  Il transumanesimo di per se’ si occupa di stabilire se e come l’umanita’ verra’ trasformata dal progresso tecnologico: il problema e’ che si focalizza molto sul corpo e sulla societa’. E’ molto difficile vedere qualcosa riguardo ai sentimenti e alla loro evoluzione. Inquinati dall’idea narrativa che i sentimenti siano imperativi, innati , eterni e assoluti per la specie umana, si crede che essi non potranno essere toccati da un cambiamento tecnologico che e’ materiale.
  
  In realta’, invece, e’ possibile che nel prossimo secolo si vada incontro ad una estinzione di massa… di sentimenti. Che poi qualche evoluzione portera’ nuovi sentimenti alla ribalta e’ da vedersi: ma che tutti i sentimenti attuali possano sopravvivere al ritmo attuale di cambiamenti sociali e culturali e’ un sogno che coltiva solo gente che vive tre metri sopra il cielo.
  
  Principalmente, scrittori di romanzetti rosa.  

\bibliography{refers}
\bibliographystyle{amsalpha}
\end{document}