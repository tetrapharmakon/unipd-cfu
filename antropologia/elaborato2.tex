\documentclass[dvipsnames]{amsart}
\usepackage{times}
\usepackage{tcolorbox}
\usepackage[italian]{babel}
\usepackage{ hyphenat, hyperref}

\usepackage[left=3.5cm,right=3.5cm]{geometry}
\tcbuselibrary{most}
\tcolorboxenvironment{quote}{blanker,
before skip=10pt,after skip=10pt,
borderline west={6mm}{5pt}{gray!15}}

\title{Antropologia filosofica}
\author{Fosco Loregian}

\begin{document}
\maketitle 

\begin{itemize}
  \item[$\bullet$] [si] discuta[no] i pregi teorici e i limiti concettuali che Plessner riscontra in un'antropologia filosofica che considera possibile una definizione della natura umana solo sul piano funzionale.
\end{itemize}
Mi concentro su un singolo dettaglio, che trovo particolarmente doloroso.

Pulendo un po' il testo da un lessico specifico abbastanza fiorito si apprezza la seguente idea: è riduttivo limitare l'uomo all'insieme delle sue prestazioni, o all'insieme delle strutture che produce materializzando i suoi processi mentali; piuttosto, un modo di affrontare il problema della definizione dell'uomo, e quindi dell'Antropologia, che sia più ``a mente vuota'' prende le mosse dall'idea che 
\begin{quote}
  l'antropologia filosofica deve considerare l'«essenza unitaria
psico-fisico-spirituale» dell'uomo e procedere quindi con un metodo di analisi che tenga conto della connessione tra processi corporei, processi di coscienza e contesti spirituali.
\end{quote}
E' lodevole, questa spinta all'unitarizzazione. Ma temo questa posizione sia solo una forma molto sottile di benaltrismo, che passa indisturbato perché è stato mascherato da un lessico impenetrabile, e successivamente condito da una vena romantico-religiosa che impedisce di attribuire alle parole un significato. Attingere a campi semantici già saturi di significato è del resto da tempo immemore un espediente per dire il nulla, nascondendosi poi dietro la suddetta sovrasaturazione.

Plessner parla nel 1964. Due anni prima è stato consegnato il Nobel a Watson e Crick, ma per idee \cite{watson1953structure} che risalivano al 1953. Undici anni non sono bastati a Plessner per familiarizzare con la genetica, per provare a farne un parametro di giudizio?

E' proprio l'esempio di Watson (e delle sue dichiarazioni leggermente razziste e omofobe in tarda età) a suggerire che l'intelligenza sia difficile da trasferire. Ma quello che si fa qui è un discorso diverso: l'efficacia di un linguaggio nel penetrare un problema destrutturandolo fino ai suoi punti fondamentali, il suo saper essere uno strumento quantitativo oltre che (o meglio, prima che) qualitativo, questo sì gerarchizza i saperi, se non nella loro dimensione morale, perlomeno nella loro capacità di analisi, nella ricchezza dei risultati che ottengono. Plessner dà la sensazione di ignorare proprio la scienza cui pretende di appoggiarsi; come dargli il minimo credito? % \emph{Ex falso quodlibet sequitur}: dovrei perdere più di un secondo dietro a delle teorie facilmente confutabili?

La necessità di un contatto più intimo con la scienza che non sia citarla a casaccio sembra sfuggire a Plessner, che ci tiene a sottolineare come ``la storia dell'antropologia filosofica moderna'' [sia] ``intrecciata alla storia delle scienze in generale'', ma poi evita con una certa cura di entrare nell'unico merito a cui la discussione scientifica avviene proficuamente, ossia quello quantitativo.

E' familiare, Plessner, con la teoria che lingua, miti, produzioni dell'intelletto siano fenomeni adattivi rispondenti a leggi simili alla selezione naturale, ma su base \emph{memetica} invece che genetica? E' familiare con l'idea che, in senso più lato, si possano applicare gli stessi modelli descrittivi\hyp{}predittivi che governano la diffusione del vaiolo e la distribuzione del calore lungo una grata alla diffusione delle idee e delle tradizioni attraverso le popolazioni umane? Questi modelli, va senza dire, afferiscono alla Matematica; non certo a Kant o Marx.

E' familiare con il circolo di idee che porterà Jaynes \cite{jaynes2000origin}, a formulare l'ipotesi -eccessiva, ma suggestiva- che la coscienza sia un sottoprodotto tardivo del linguaggio? Con le immediate ed estreme conseguenze che l'autocoscienza non è più una caratteristica definiente dell'uomo, e quindi è fallace pensare i primi uomini -degli uomini che vissero gli eventi che ispirarono l'Iliade- abitati dagli stessi moti dello spirito di cui disponiamo noi? 

E' familiare col fatto che se c'è questa frattura, non c'è continuità alcuna tra ``l'uomo'' come fu inteso dai classici e l'uomo come è inteso dai moderni? Plessner lo ha letto,``Il crollo della mente bicamerale'',e se sì, che ne pensa?

E' familiare con la teoria delle strutture emergenti, coi fenomeni intrinsecamente caotici (cioè con la dipendenza caotica di un sistema dinamico dai suoi dati iniziali così come studiata da Lyapunov -che muore nel 1918: Plessner e i suoi compari avrebbero avuto tutto il tempo di aprire un libro di meccanica dei sistemi\dots).

E' familiare con l'analisi delle reti sociali \cite{10.2307/2785588} che nasce, almeno in nuce, con Moreno e Jennings negli anni '30? E' da notare che Moreno e Jennings furono influenzati da Durkheim; non esattamente un estraneo ai congeneri di Plessner. 

Supponiamo che questa familiarità ci sia: queste competenze non traspaiono. Ma più probabilmente no, Plessner non frequenta questi salotti né questa maniera di pensare. E questo non sarebbe un problema, se solo la parola ``scienza'' non occorresse abbastanza spesso da far prurito. 

\emph{Bisogna indagare il rapporto tra le scienze della natura e le scienze della cultura; la storia dell'antropologia filosofica moderna è intrecciata alla storia delle scienze in generale; l'Antropologia filosofica deve porsi al punto medio tra due tensioni distinte, la spinta al generale} (prerogativa delle scienze sociali -frase assai opinabile, a esser gentili) \emph{e lo studio del particolare} (che sarebbe invece prerogativa delle scienze esatte? E' uno scherzo?).

Le hanno studiate, mi chiedo io, quelle scienze a cui ambiscono a intrecciarsi, o piuttosto non hanno la minima idea di cosa le componga quotidianamente? Hanno la minima familiarità con la contaminazione tra scienza e strutturalismo, negli stessi anni '60 in cui Plessner definisce baroccamente l'uomo come `orizzonte dei compiti di un essere che desidera e spera eccetera eccetera' (benissimo: ma allora cos'è un \emph{orizzonte}, e qual è la definizione operativa di \emph{desiderio}, e qual è il motivo per cui questa definizione non è mal posta, di quale linguaggio formale fa parte, cosa assicura che questa definizione non sia mal-tipata\dots)? Cos'è una società se non un sistema dinamico, e che cos'è questa fumosa e tanto sbandierata paura che la ricerca sia ridotta a categorie glaciali, prive di sensibilità olistica, se non ignoranza della teoria dei sistemi emergenti? Il concetto per cui il tutto è maggiore della somma delle sue parti \emph{è matematizzabile}: ignorare questo fatto significa temere il linguaggio scientifico come il più sfaticato dei ragazzini.

Difficilmente si scusa l'ignoranza di Plessner se non sospettandone la pigrizia intellettuale, figlia di un atteggiamento castale comune tra gli umanisti, attaccati pervicacemente al desiderio di evitare qualsiasi riferimento a parametri misurabili, oggettivi (non fatico a crederlo, è ciò che ogni gruppo sociale fa quando può scegliere tra l'autodifesa e l'onestà: un minimo approccio quantitativo distruggerebbe alla radice molte di quegli artifici retorici che hanno fatto la carriera di tanti ``pensatori'' che ambiscono a penetrare le intime fondamenta dell'Essere e poi sono in imbarazzo a fare una derivata). La corrente ``filo\hyp{}scientifica'' di questi antropologi si può invalidare appellandosi a un meccanismo comprensibile con modeste conoscenze di teoria dei sistemi interagenti (ogni gruppo sociale predilige l'auto\hyp{}conservazione ad un'analisi critica potenzialmente distruttiva), facendo sì che i propalatori di questa pseudo-teoria siano impossibilitati, dalla propria ignoranza scientifica, ad apprezzare i limiti del loro pensiero: alla luce di questa beffa, cosa dovremmo pensare, dell'antropologia filosofica come la intendono i padri fondatori?

La scarsa frequentazione dello stato dell'arte in una teoria che sta andando definendosi mentre Plessner scriveva è probabilmente perdonabile da uno spirito meno rigido del mio, che crede gli sarebbe bastato aprire un libro o due in più. Ma l'evitare \emph{chirurgicamente} ogni tentativo anche minimo di spostare la discussione sul piano quantitativo, persino di quei tentativi che sono palesemente mutuati dai lavori di sociologi, non può non puzzare, da molto lontano, di disonestà intellettuale, pressapochismo, e in poche parole di quella perniciosa visione del sapere tardoromantica che tiene in scacco il progresso e la contaminazione tra i saperi, impedendo a scienza e filosofia di essere per davvero due sorelle ingiustamente divise in culla.

Concludo: il poco che sappiamo a proposito dei meccanismi che regolano l'interazione tra decisori razionali lo sappiamo perché disponiamo della teoria dei giochi, della teoria dei sistemi complessi, della teoria dei fenomeni caotici e dei sistemi emergenti. Sicuramente non lo sappiamo per merito di qualcuno la cui migliore definizione di uomo è ``l'orizzonte dei compiti di un essere che desidera e spera eccetera eccetera''. E' questo il distillato migliore che la ``ricerca'' riesce a proporre come definizione dell'oggetto di studio di una disciplina? No: questo, ai miei occhi, è un suppurato. Non è solo irrilevante: è dannoso, è un'infezione che è penetrata nel midollo. %Questo approccio alla conoscenza che pretende a scarso diritto di essere accostata alla scienza, senza averla praticata un giorno, non è nemmeno la parodia di un sistema di pensiero: se non altro, perché la parola \emph{parodia} ha un fondo di dignità storica ed etimologica che a queste idee manca del tutto.

L'impressione, purtroppo, è sempre la stessa: la filosofia è quella malattia dello spirito che ha fatto un caso di stato della domanda ``domani il sole sorgerà o no?''. La scienza è quella cosa che a un certo punto ha guardato in alto e ha chiesto ``andiamo sulla Luna?''.

\bibliography{refers}
\bibliographystyle{amsalpha}
\end{document}